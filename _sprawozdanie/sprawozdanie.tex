\documentclass[a4paper,11pt,titlepage]{article}

\usepackage{latexsym}
\usepackage{graphicx}
\usepackage{float}
\usepackage{url}
\usepackage{unicode}
\usepackage[polish]{babel}
\usepackage{titlesec}
\usepackage{listings}
\usepackage{xcolor}
\usepackage{setspace}
\usepackage{subfig}
\usepackage{tabularx}
\usepackage{courier}
\DeclareUnicodeCharacter{200B}{{\hskip 0pt}}

\definecolor{codeblue}{rgb}{0,0,0.6}
\definecolor{codegray}{rgb}{0.5,0.5,0.5}
\definecolor{codepurple}{rgb}{0.58,0,0.82}
\definecolor{backcolour}{rgb}{0.96,0.96,0.96}

\lstdefinestyle{code}{
    backgroundcolor=\color{backcolour},
    keywordstyle=\color{codeblue},
    numberstyle=\tiny\color{codegray},
    stringstyle=\color{codeblue},
    basicstyle=\ttfamily\footnotesize,
    breakatwhitespace=false,
    breaklines=true,
    captionpos=b,
    keepspaces=true,
    numbers=left,
    numbersep=3pt,
    showspaces=false,
    showstringspaces=false,
    showtabs=false,
    tabsize=1,
    basicstyle=\small
}

\lstset{style=code}

\newcommand{\sectionbreak}{\clearpage}
\author{Adam Talarczyk}
\title{Symulacja Wieloagentowa}
\frenchspacing
\begin{document}
\begin{titlepage}
    \begin{center}

        \Huge
        \textbf{WYDZIAŁ NAUK ŚCISŁYCH I TECHNICZNYCH}
        
        
        \vspace{1.5cm}
	   Symulacje Komputerowe
        \LARGE
        
	\vspace{2cm}
	
	Sprawozdanie ``Symulacja Wieloagentowa''

	\vspace{1cm}
	Adam Talarczyk, Mateusz Wrzoł
	
	\vspace{5cm}
        \vfill

        \vspace{0.8cm}
	\Large
        Uniwersytet Śląski, Sosnowiec, 2021

    \end{center}
\end{titlepage}
\newpage

\tableofcontents
\newpage

\section{Zadanie 1}
Należy opracować symulator dowolnego zjawiska lub procesu, wykorzystując model wieloagentowy.

Symulator powinien być wyposażony następujące funkcje:
\begin{itemize}
\item wizualizacja stanu środowiska i agentów,
\item wykres(y) z wynikami symulacji,
\item interfejs użytkownika umożliwiający modyfikowanie parametrów modelu.
\end{itemize}


Sprawozdanie powinno zawierać:
\begin{itemize}
\item opis zaimplementowanego modelu wieloagentowego,
\item kod źródłowy symulatora z komentarzami,
\item prezentację interfejsu użytkownika z zrzutami ekranu,
\item przykładowe wyniki symulacji,
\item spis bibliografii (jeżeli była wykorzystana).
\end{itemize}

Dodatkowo poza sprawozdaniem proszę przesłać pik(i) z projektem symulatora (np. plik Netlogo).

Ocena rozwiązania będzie uwzględniała:
\begin{itemize}
\item stopień skomplikowania zaproponowanego modelu i opracowanego symulatora,
\item oryginalność rozwiązania (symulator nie może być prostą modyfikacją modeli symulacyjnych dostępnych w Netlogo lub innego gotowego oprogramowania),
\item jakość przygotowanego sprawozdania.
\end{itemize}

Do rozwiązania zadania można wykorzystać środowisko Netlogo lub dowolne inne środowisko programistyczne.

\end{document}
